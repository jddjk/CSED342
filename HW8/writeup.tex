\documentclass[12pt]{article}
\usepackage{fullpage,enumitem,amsmath,amssymb,graphicx}

\begin{document}

\begin{center}
{\Large CSED342 Assignment 8 \vspace{10pt}}

\begin{tabular}{rl}
Student ID: & 20220036 \\
Name: & Jaehyeon Lee \\
\end{tabular}
\end{center}

\begin{center}
By turning in this assignment, I agree by the POSTECH honor code and declare that all of this is my own work.
\end{center}

\section*{Problem 2a}

\subsection*{Knowledge Base in CNF}
First, we convert each formula in the knowledge base \( KB = \{(A \lor B) \rightarrow \neg C, \neg(\neg A \lor C) \rightarrow D, A\} \) into CNF.

\begin{enumerate}
    \item Convert \((A \lor B) \rightarrow \neg C\):
    \[
    (A \lor B) \rightarrow \neg C \equiv \neg (A \lor B) \lor \neg C.
    \]
    Using De Morgan's laws:
    \[
    \neg (A \lor B) \equiv \neg A \land \neg B.
    \]
    Thus:
    \[
    \neg (A \lor B) \lor \neg C \equiv (\neg A \land \neg B) \lor \neg C \equiv (\neg A \lor \neg C) \land (\neg B \lor \neg C).
    \]
    \item Convert \(\neg(\neg A \lor C) \rightarrow D\):
    \[
    \neg(\neg A \lor C) \rightarrow D \equiv \neg \neg(\neg A \lor C) \lor D \equiv (\neg A \lor C) \lor D \equiv \neg A \lor C \lor D
    \]
    \item The formula \(A\) is already in CNF.
\end{enumerate}

Thus, the CNF form of the knowledge base is:
\[
KB = \{\neg A \lor \neg C, \neg B \lor \neg C, \neg A \lor C \lor D, A\}.
\]

\subsection*{Derivation using Modus Ponens}
Now, we use Modus Ponens to derive \(D\).

\begin{enumerate}
    \item From \(A\), apply to \(\neg A \lor \neg C\):
    \[
    \frac{A, \neg A \lor \neg C}{\neg C} \quad \text{(Modus Ponens)}
    \]
    \item From \(\neg C\), apply to \(\neg A \lor C \lor D\):
    \[
    \frac{\neg C, C \lor \neg A \lor D}{\neg A \lor D} \quad \text{(Modus Ponens)}
    \]
    \item From \(A\), apply to \(\neg A \lor D\):
    \[
    \frac{A, \neg A \lor D}{D} \quad \text{(Modus Ponens)}
    \]
\end{enumerate}

Thus, we derive \(D\).

\section*{Problem 2b}

\subsection*{Knowledge Base in CNF}
Convert the knowledge base \(KB = \{A \lor B, B \rightarrow C, (A \lor C) \rightarrow D\}\) into CNF.

\begin{enumerate}
    \item The formula \(A \lor B\) is already in CNF.
    \item Convert \(B \rightarrow C\):
    \[
    B \rightarrow C \equiv \neg B \lor C.
    \]
    \item Convert \((A \lor C) \rightarrow D\):
    \[
    (A \lor C) \rightarrow D \equiv \neg (A \lor C) \lor D \equiv (\neg A \land \neg C) \lor D \equiv (\neg A \lor D) \land (\neg C \lor D).
    \]
\end{enumerate}

Thus, the CNF of the knowledge base is:
\[
KB = \{A \lor B, \neg B \lor C, \neg A \lor D, \neg C \lor D\}.
\]

\subsection*{Derivation using the Resolution Rule}
We use the resolution rule to derive \(D\).

\begin{enumerate}
    \item Resolve \(A \lor B\) and \(\neg B \lor C\):
    \[
    \frac{A \lor B, \neg B \lor C}{A \lor C} \quad \text{(Resolution)}
    \]
    \item Resolve \(A \lor C\) and \(\neg A \lor D\):
    \[
    \frac{C \lor A, \neg A \lor D}{C \lor D} \quad \text{(Resolution)}
    \]
    \item Resolve \(C \lor D\) and \(\neg C \lor D\):
    \[
    \frac{D \lor C, \neg C \lor D}{D} \quad \text{(Resolution)}
    \]
\end{enumerate}

Thus, we derive \(D\).

\end{document}